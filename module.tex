\documentclass[headings=small,cmyk,french,a4paper,twocolumn,garamond,11pt]{scrartcl}
\usepackage[total={177mm,260mm},marginparwidth=0pt,marginparsep=0pt,showframe=false]{geometry}\usepackage{scrjura}
\usepackage{babel}
\usepackage{glossaries}
\usepackage[dvipsnames,cmyk]{xcolor}
\usepackage{microtype}
% \AddtoOneTimeDoHook{heading/begingroup}{\setheadingcolor{#1}}
% \newcommand*{\setheadingcolor}[2]{\addtokomafont{#2}{\color{#1}}}
\addtokomafont{section}{\color{BrickRed}}
\usepackage{datatool}
\usepackage{environ}
\usepackage{hyperref}
\usepackage{pdflscape}
\DTLnewdb{exigences}
\newcounter{toto}[subsection]

%\usepackage{glossaries}
%\usepackage{xcolor}
\newacronym{ANSSI}{\textsc{anssi}}{agence nationale de sécurité des systèmes d'information}
\newacronym{BD}{\textsc{bd}}{Bases de données}
\newacronym{BIATS}{\textsc{biats}}{Bibliothèque, ingénieur administratif, technique}
\newacronym{BLE}{\textsc{ble}}{Bluetooth low energy}
\newacronym{CCAG}{\textsc{ccag}}{Cahier des clauses administratives générales}
\newacronym{CESREES}{\textsc{cesrees}}{Comité éthique et scientifique pour les recherches, les études et les évaluations dans le domaine de la santé}
\newacronym{CEV}{\textsc{cev}}{Commission d'évaluation des voeux d'admission}
\newacronym{CNIL}{\textsc{cnil}}{Commission nationale informatique et libertés}
\newacronym{CER}{\textsc{cer}}{Comité d'éthique de la recherche}
\newacronym{CPP}{\textsc{cpp}}{Comité de protection des personnes}
\newacronym{CSIESR}{\textsc{csiesr}}{comité des Services Informatiques Enseignement Supérieur et Recherche}
\newacronym{DCP}{\textsc{dcp}}{donnée à caractère personnel}
\newacronym{DPO}{\textsc{dpo}}{délégué à la protection des données}
\newacronym{DSIUN}{\textsc{dsiun}}{\textsc{dsiun}}
\newacronym{ENT}{\textsc{ent}}{Environnement numérique de travail}
\newacronym{ESR}{\textsc{esri}}{dnseignement supérieur et recherche}
\newacronym{ECHR}{\textsc{cedh}}{charte européenne des droits de l'homme}
\newacronym{EPRIV}{e\textsc{P}rivacy}{Directive}
\newacronym{HFDS}{\textsc{hdfs}}{haut fonctionnaire défense et sécurité}
\newacronym{HCERES}{\textsc{hceres}}{\textcolor{red}{hceres}}
\newacronym{IGESR}{\textsc{igésr}}{Inspection générale éducation sport et recherche}
\newacronym{INUC}{\textsc{inuc}}{Institut National Universitaire Champollion}
\newacronym{LIL}{\textsc{lil}}{Loi relative à l'informatique, aux fichiers et aux libertés}
\newacronym{MR}{\textsc{mr}}{Méthode de référence CNIL}
\newacronym{MESR}{\textsc{mesri}}{Ministère de l'enseignement supérieur, de la recherche et de l'innovation}
\newacronym{NREN}{\textsc{nren}}{Réseau national de recherche et de formation}
\newacronym{ORCID}{\textsc{orcid}}{\textcolor{red}{orcid}}
\newacronym{PDME}{\textsc{pdme}}{plan de mobilité employeur}
\newacronym{PDS}{\textsc{pds}}{Plateforme des données de santé/Health data hub}
\newacronym{PSS}{\textsc{pss}}{Pôle santé social de la vie étudiante}
\newacronym{RENATER}{\textsc{renater}}{Réseau national enseignemet recherche}

\newacronym{RGPD}{\textsc{rgpd}}{Règlement relatif à la protection des
    personnes physiques à l'égard du traitement des données à
    caractère personnel et à la libre circulation de ces données}
\newacronym{RH}{\textsc{rh}}{Ressources humaines}
\newacronym{RIPH}{\textsc{riph}}{Recherche impliquant la personne humaine}
\newacronym{RSSI}{\textsc{rssi}}{Responsable de la sécurité des systèmes d'information}
\newacronym{RT}{\textsc{rt}}{responsable du traitement}
\newacronym[description=\href{https://welcomedesk.univ-toulouse.fr/vie-quotidienne/sante-social/simpps-service-medical-social-dedie-aux-etudiants}{Information SIMPPS}]{SIMPPS}{\textsc{simpps}}{Service interuniversitaire de médecine préventive }
\newacronym{TFEU}{\textsc{tfeu}}{Traité de fonctionnement de l'union}
\newacronym{UT}{\textsc{ut}}{Université de Toulouse}
\newacronym{UrbaESR}{\textsc{u}rba\texttt{esr}}{gpe urbanistes de l'\gls{ESR}}
\newacronym{USB}{\textsc{usb}}{Universal Serial Bus}
\newacronym{UWB}{\textsc{uwb}}{Ultra Wide Band}
\newacronym{VPN}{\textsc{vpn}}{Virtual private network}
\newacronym{ZRR}{\textsc{zrr}}{zone à régime restrictif}

%%% Local Variables:
%%% mode: latex
%%% TeX-master: "memo"
%%% End:

\title{Module sécurité : balises de traçage radios}\date{}
\begin{document}
\maketitle{}
\section{Description}
\subsection{Introduction}
Les balises de traçage radios (tracker tags)  sont des objets connectés
autonomes qui, liés à des biens matériels, promettent à leurs
propriétaires d'en éviter l'oubli ou d'en faciliter les recherches en
cas de perte ou de vol.

Les capacités de ces balises reposent sur l'utilisation de connexions
radio (\acrshort{BLE},  \acrshort{UWB}) entre la balise et un
\emph{équipement mobile maître} (téléphone, tablette) détenu par le  propriétaire.
% permettent de suivre par radio, sur une distance moyenne,
% la localisation d'un bien. La liaison entre la balise est le bien
% tracé s'effectue généralement par collage, lien d'attache ou
% porte-clefs.

% Cette capacité de localisation radio est destinée à éviter l'oubli du
% bien et à faciliter les recherches en cas de disparition (perte ou vol) de
% ce bien.
\paragraph{Fonctionnement de proximité}
Réalisées par différents constructeurs, ces balises reposent sur un
principe similaire de fonctionnement : une balise est appariée par
protocole radio à un ou plusieurs \emph{équipements mobiles maîtres} dont
elles doivent rester proches. La balise est \textsl{«silencieuse»}
tant qu'elle est connectée à l'un de ses équipements maîtres.  Lorsque
la balise n'est plus connectée, elle diffuse publiquement sa situation
afin de pouvoir être facilement localisée.

Le fonctionnement de proximité des systèmes de balise repose donc sur
l'installation d'une application dédiée sur l'\emph{équipement maître}. Cette
application permet initialement de configurer la balise, puis en état
de marche, de gérer la connexion radio du mobile avec la balise,
d'afficher les notifications et enfin de guider le propriétaire en cas
de recherche.
\paragraph{Fonctionnement à grande échelle}
Le fonctionnement de proximité est complété par une capacité de
recherche à grande échelle : la diffusion publique par la balise de sa
situation permet à de multiples équipements tiers compatibles de la
localiser. Ces équipements tiers sont alors en mesure de propager la
situation de la balise à un serveur distant opéré par le
constructeur. Ce mode de localisation, désigné par l'expression
«localisation participative» (\emph{crowdsourced network}) peut
permettre au propriétaire de retrouver le bien égaré ou volé même s'il
n'est plus dans la portée radio de ses équipements maîtres.

La mise en oeuvre de la capacité de recherche à grande échelle repose
non seulement sur l'existence de liens radio de proximité ainsi que
d'une application dédiée de gestion sur le téléphone, mais aussi sur
la mise à disposition, par le fournisseur de la balise, d'un service
distant de suivi de position, éventuellement présenté sous forme
cartographique.

Les différents systèmes de localisation ne sont pas, en général,
intéropérables: le systèmes de suivi à grande échelle d'un vendeur
n'est pas compatible avec ceux d'autres vendeurs.


% Les balises ne sont pas des dispositifs d'intérêt en elles-mêmes mais
% au travers de l'accessoire auquel elles sont associées.
\subsection{Objectifs}

\subsection{Périmètre du module}
Ce module ne s'applique qu'aux balises \gls{BLE} dont l'identification
est restreinte tant qu'elles sont dans la portée d'une plate-forme
maître.

En particulier, ce module ne s'applique pas pas aux balises de
localisation utilisées dans un cadre logistique ni aux balises
autonomes qui possèdent des capacités de traitement élevées (GPS,
connexion réseau étendue) ni aux systèmes radio de courte portée tels
que les système NFC ou RFID.

% La brique «Balises de traçage radios» devrait être complétée par
% différentes briques de sécurité : «équipements mobiles», «équipements
% IOT», «gestion des matériels à visée domestique».
\section{Panorama de la menace}
\subsection{Divulgation de finalité}
L'appariement d'une balise s'effectue en associant une description 
du bien tracé, par exemple \textsl{«Télécommande de pilotage du
  système antivol»}. En cas de divulgation, cette désignation pourrait
réveler des informations essentielles sur la valeur du bien.

\subsection{Divulgation  de position}
Les biens associés à une balise peuvent être parfois déposés
dans un endroit considéré comme «sûr» en raison de son emplacement
confidentiel.  Si l'équipement maître de la balise n'est pas déposé en
proximité, la balise va signaler sa position et pourrait ainsi révèler
l'existence et la position du lieu de dépôt.
% qLe personnel affectataire de l'accessoire tracé doit être en
% possession de'un équipement mobile d'appariement afin qu'il ne soit
% pas continuement sous surveillance, il ne suffit pas d'apparier la
% balise avec le téléphone d'un gestionnaire.

\subsection{Intrusivité des applications de gestion}
Une balise n'embarque pas de dispositif GPS, elle ne possède donc
aucune capacité de localisation géographique. Ces capacités
géographiques  sont fournies par l'équipement maître auquelle elle
est est appariée.

La réalisation de la capacité de localisation au sein du système
repose donc sur l'activation, plus ou moins fine, des fonctions de
géolocalisation sur les équipements maîtres.

De plus, l'activation d'autres fonction plus ou moins intrusives
peuvent aussi être nécessaires au fonctionnement de l'application.

\subsection{Activation des protocoles radios}

La fonction d'appariement des matériels maîtres exige que le protocole
\acrshort{BLE} soit systématiquement activé sur les équipements maîtres.
Comme les protocoles de la famille \acrshort{BLE} font périodiquement
l'objet de divulgation de vulnérabilités, cette activation augmente
considérablement la surface d'attaque des matériels concernés.

\subsection{Consommation de ressources}
Les balises sont conçues pour être économes en énergie.  Ce n'est
pas le cas pour les appareils maîtres qui, lorsque le protocole
\acrshort{BLE} et la géolocalisation sont activés, doivent répondre à de
nombreuses sollicitations. L'autonomie des appareils maîtres sera donc
diminuée dès la mise en place d'une association avec une nouvelle
balise.

\subsection{Persistence  de l'identifiant réseau}
Les identifiants réseau utilisés par les balises pour leurs connexions
réseau peuvent être persistants ou aléatoires. Dans le premier cas, la
persistance expose la balise à la possibilité d'un suivi continu de sa
localisation et de celle de son porteur.
\subsection{Déficiences de fonctionnement}
Le suivi à grande échelle pourraient être mis en défaut si le nombre
de participants au réseau participatif n'est pas assez élevé ou si les
détenteurs de plate-formes compatibles n'ont pas activé les fonctions
adéquates sur leurs équipements mobiles.

La détention d'une balise et de son application de gestion peuvent ne
pas suffire pas à disposer de toutes les fonctions, fonctions
supplémentaires qui ne sont activées que sur contrat de souscription.



\subsection{Maintien en condition de sécurité}
Le logiciel embarqué dans les balises \acrshort{BLE} possède des capacités
limités. Ce logiciel est cependant susceptible de comporter des
défauts ou bien de devoir être modifié pour suivre les évolutions des
protocoles de gestion. Afin d'être maintenu en condition de
sécurité, le logiciel embarqué des balises doit pouvoir être mis à
jour afin de suivre ses évolutions du constructeur.

\subsection{Absence, par conception, de mesures anti vie privée }
Les balises de traçage sont parfois utilisée pour réaliser un suivi
abusif de personnes (harcèlement) , comme en témoignent plusieurs affaires
judiciaires.

En réaction, pour que les victimes potentielles puissent constater un
traçage abusif, les constructeurs facilitent volontairement la
détection de balises par des tierces personnes.

L'existence de ces mesures est en contradiction avec les principes de
préservation des droits et des libertés fondamentales au regard de la
vie privée et des traitements de données à caractère personnel.

\subsection{Difficulté de multipropriété}
La plupart des fournisseurs associent ces balises à un compte
identifié sur leurs serveurs. Cette identification fine du
propriétaire permet à la fois de déclarer automatiquement les
plate-formes maîtres de la balise et de faciliter l'identification du
responsable d'un suivi abusif de personne.

Cependant, dans un cadre organisationnel, ces mesures ne permettent pas
de discriminer finement les personnes qui ont accès au service distant
de gestion . Ces mesures peuvent aussi poser des problèmes de
couverture si l'ensemble des  équipements de l'organisation sont
enregistrées avec le même identifiant client.


\subsection{Suivi continu des personnes affectataires}
Une balise diffuse publiquement sa situation lorsqu'elle est hors de
portée de ses équipements pairs. Cette diffusion peut mener à une
suivi systématique de la localisation du bien et de son
porteur. Le suivi systématique d'un personnel, d'un usager ou d'une
tierce pourrait poser des problèmes juridiques.
% \subsection{DCP  des gestionnaires}
% L'interférence des applications de gestion (application téléphone,
% application serveur) doit être prise en compte dans le cadre du
% processus de sélection des balises.

\subsection{Défaut inattendu d'appariement }
Une balise signale sa situation après un certain délai lorsque'elle
n'est plus connectée à son équipemet maître. Ce défaut d'appariement
peut se produire lorsque l'équipement pair ne dispose plus d'énergie ou
lorsqu'il doit faire l'objet d'une opération de maintenance.

\subsection{Détournement de finalité}
Une balise pourrait être utilisée à des fins de traçage de bien ou de
personnes en dehors de tout besoin organisationnel.

\section{Exigences}
Cette section décrit les exigences de sécurité relative au déploiement
de balises de traçage au sein de l'établissement. Le RSSI est chargé
de vérifier que ces exigences sont satifaites et que leur mise en
eouvre est cohérente avec les objectifs de sécurité.

Les rôles et responsabilités relatifs à ce module sont les
suivants :
\begin{description}
\item[responsable métier] personne responsable du service ou de
  l'unité organisationnelle en charge de l'affectation et du suivi des
  balises;
\item[affectataire] personne (personnel, usager, sous-traitant) à
  laquelle est affecté un bien associé à une balise;
\item[gestionnaire technique] service en charge de la vérification du
  fonctionnement des types de balise et du maintien en condition de
  sécurité des équipments mobiles maîtres de la balise
\end{description}
\stepcounter{toto}%\arabic{toto}
\newcommand{\dbadd}[1]{
  \stepcounter{toto}%
  \DTLnewrow{exigences}%
  \dtlexpandnewvalue%
  \DTLnewdbentry{exigences}{Id}{E--\arabic{subsection}.\arabic{toto}}%
  \DTLnewdbentry{exigences}{Réf.}{\ref{E.\arabic{subsection}}}%
  \DTLnewdbentry{exigences}{Question}{#1}%
  \DTLnewdbentry{exigences}{N/A}{\fbox{~}}%
  \DTLnewdbentry{exigences}{Oui/Non}{\fbox{~}}%
}





\let\oldsubsection\subsection
\renewcommand{\subsection}[2][]{\oldsubsection[#1]{#2}\label{E.\arabic{subsection}}}
%\renewcommand{\subsection}[1]{\oldsubsection}
\subsection{Centralisation de  gestion}
\dbadd{La gestion des balises de traçage est centralisée}
\dbadd{Les agents gestionnaire du suivi sont habilités}
\dbadd{Les accès au serveur de gestion des balises sont individuels}
% \DTLnewrow{exigences}
% \DTLnewdbentry{exigences}{Id}{E.\thesubsection}
% \DTLnewdbentry{exigences}{Question}{La gestion des balises de tracage est  centralisée}
Un seul service organisationnel doit être responsable de la
configuration initiale, de l'affectation et du suivi des balises;

Les agents en charge du suivi des balises doivent être formellement
habilités à le faire. Le compte de gestion de la composante serveur du
système de localisation ne devrait pas être mutualisé, chaque accès
devrait être personnel.
\subsection{Maintien en condition de sécurité}
\dbadd{Le maintien en condition de sécurité est prévu dans le contrat ou dans les CGU}
\dbadd{Le firmware des balises peut être mis-à-jour}
\dbadd{Une page web dédiée à la sécurité des balises est hébergée par le fournisseur}

Les équipements acquis doivent pouvoir être mis à jour en cas de
défaut de sécurité. Cette possibilité de mise à jour doit être
explicitement décrite dans le contrat sépcifique d'acquisition ou les
conditions générales de vente.

Une page web dédiée à la sécurité des balises et de leur logiciel de
gestion devrait être mise à disposition par le vendeur.

\subsection{Utilisation de matériels maîtres professionnels}
\dbadd{Les équipements maîtres de la balise ne sont pas des matériels personnels}

Les téléphones, les ordinateurs ou les tablettes maîtres de la balise
doivent être des matériels professionnels gérés par l'établissement,
les applications de gestion ne doivent pas être installées sur des
téléphones domestiques (personnels).
\subsection{Intrusivité des applications mobiles de gestion}
\dbadd{L'intrusivité de l'application de gestion mobile des balises a été évaluée}

L'intrusivité des applications de gestion --- trackers, périmètre des
autorisation exigées, interaction avec d'autres applications, couplage
avec le système d'exploitation --- devrait être prise en compte dans les
critères de choix de la solution.

\subsection{Intrusivité des applications serveurs}
\dbadd{L'intrusivité de l'application de gestion serveur des balises a été évaluée}

L'intrusivité des applications serveurs relativement aux droits et
libertés des personnels en charge de la gestion des balise devrait
être évaluée et prise en compte dans les critères de choix de la
solution.
\subsection{Utilisation d'une adresse mail dédiée pour la gestion}
\dbadd{L'adresse mail de contact associée aux balises est une adresse dédiée}

Si le renseignement d'une adresse mail est demandée lors du
paramétrage de la balise, une adresse mail fonctionnelle dédiée à la
gestion des balises doit être utilisée.

\subsection{Insignifiance des descriptions}
\dbadd{La description des balises est un texte neutre qui ne dévoile pas la finalité du bien associé}

Les descriptions des biens tracés ne devraient pas posséder de sens
précis afin qu'une divulgation de ces descriptions ne permette pas à
une tierce partie de déduire l'utilisation qu'il peut faire du bien
tracé.

\subsection{Sécurisation des appariements en cas de défaut d'alimentation}
\dbadd{Une batterie de secours pour l'équipement maître a été adffectée en même temps que celui-ci}

Les personnes affectataires devraient pouvoir recharger l'équipement
maître à tout moment afin d'éviter un désappariement. En raison de
l'augmentation de la consommation due à l'activation du \acrshort{BLE}
et de la géolocalisation, une batterie de secours devrait être mise à
leur disposition en même temps que les autres matériels.

\subsection{Sécurisation les appariements en cas de maintenance}
\dbadd{Le désappariement de la balise a été prévu en cas de maintenance de l'équipement maître}

Les activités de maintenance doivent prendre en compte la possibilité
d'un désappariement de la balise, le déroulé de ces activités devrait
être formalisé.

\subsection{Sensibilité des lieux de lieux de stockage de biens}
\dbadd{Le stockage de la balise ne risque pas de dévoiler la localisation d'un lieu de dépôt confidentiel}

Le cas échéant, la sécurisation/sensibilité des stockages temporaires
de biens (boîte à clés de bâtiment, \ldots) suivis par balise doit
être étudiée afin de déterminer si ces balises doivent être
désactivées lors de ces dépôts et/ou si le lieu considéré doit faire
l'objet d'une sécurisation supplémentaire.


\subsection{Procédure de gestion des pertes de bien}
\dbadd{Les actions a entreprendre en cas de perte de bien ont été prévues}

Une procédure de traitement doit être définie en cas de signalisation
de désappariement.  Cette procédure doit engager le personnel
affectataire de la balise qui doit être informé de son déroulement.

\subsection{Désactivation de  balise}
\dbadd{Les conditions de désactivation de la balise sur demande de l'affectataire sont prévues}

Les gestionnaires en charge des balises devrait pouvoir être en mesure
de désactiver une balise sur demande motivée de l'affectataire.

\onecolumn{}
% \renewcommand{\dtlbeforecols}{\protect\hline}
\renewcommand{\dtlbeforecols}{|}
\renewcommand{\dtldisplaystarttab}{\hline}
\renewcommand{\dtldisplayafterhead}{\hline}
\begin{landscape}
  \section*{Évaluation}
  \DTLdisplaydb{exigences}
  \section*{Remarques}
\end{landscape}

%\section{Menaces élémentaires}
%\begin{itemize}
%\item Défaut ou interruption d'alimentation électrique
%\item Défaut ou interruption de réseau de communication
%\item Défaut ou interruption d'un fournisseur de service
% \item Interférence électro-magnétique
% \item Interception ou compromission  d'interférence
%\item Divulgation d'information, espionnage
% \item Écoute discrète
%\item Vol d'équipement, de périphérique de stockage ou de documents
%\item Perte  d'équipement, de périphérique de stockage ou de documents
%\item Défaut de planification ou manque d'adaptation
%\item Divulgation d'informations sensibles
% \item Source non fiable d'information ou de produits
% \item Falsification logicielle ou matérielle
% \item Falsification d'informations
% \item Accès abusif à un système informatique
% \item Destruction de matériels ou de périphérique de stockage
% \item Panne matérielle ou système
% \item Dysfonctionnement matériel ou système
% \item Défaut de ressources
% \item Vulnérabilité ou erreurs logicielles
% \item Violation de loi ou de réglement
% \item Utilisation abusive ou accès administrateur à un matériel ou un système
% \item Utilisation incorrecte d'un matériel ou d'un systèmùe
% \item Abus d'autorisation
% \item Défaut de personnel
% \item Attaque
% \item Coercition, chantage ou corruption
% \item Usurpation d'identité
% \item Répudiation d'action
% \item Abus d'utilisation d'information à caractère personnel
% \item Logiciel malicieux 
% \item Dénis de service
% \item Sabotage
% \item Ingéniérie sociale
% \item Attaque par falsification de messages
% \item Passage non autorisé
% \item Perte de données
% \item Perte d'intégrité d'informations sensibles
% \item Effets secondaire nocifs d'une attaque orientée SI

%\end{itemize}


\end{document}



%%% Local Variables:
%%% TeX-parse-self: t
%%% TeX-auto-save: t
%%% TeX-engine: luatex
%%% End: 
